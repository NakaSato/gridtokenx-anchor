\chapter{Conclusion}
\label{chap:conclusion}

\section{Summary of Contributions}

This thesis makes the following contributions to the field of blockchain-based energy trading:

\begin{enumerate}
  \item \textbf{Performance Evaluation Framework}: We adapted TPC benchmarks (TPC-C, TPC-E, TPC-H) for blockchain evaluation, providing a rigorous methodology for assessing distributed ledger performance in energy trading applications.
  
  \item \textbf{GridTokenX Platform}: We developed and evaluated a Solana-based blockchain platform for P2P energy trading using Proof of Authority consensus, demonstrating production-level performance characteristics.
  
  \item \textbf{Quantified Trust Premium}: We introduced and measured the Trust Premium metric, quantifying the performance overhead of decentralization at 5.65x compared to centralized databases.
  
  \item \textbf{Scalability Analysis}: We demonstrated linear scalability up to 200 concurrent users with maintained sub-20ms latency, validating the platform's suitability for microgrid deployments.
\end{enumerate}

\section{Key Findings}

\subsection{Performance Achievement}

GridTokenX achieves 21,101 tpmC on TPC-C style workloads, demonstrating OLTP capability sufficient for real-time energy trading. The sub-20ms p99 latency meets the requirements for automated demand response and real-time pricing applications.

\subsection{Comparative Advantage}

Compared to existing blockchain platforms:
\begin{itemize}
  \item 10x lower latency than Hyperledger Fabric
  \item 600x lower latency than Ethereum
  \item Acceptable 5.65x overhead versus centralized solutions
\end{itemize}

\section{Limitations}

This research has the following limitations:

\begin{enumerate}
  \item Benchmarks conducted on simulated network conditions
  \item Single-validator PoA configuration, not full production network
  \item Limited geographic distribution testing
  \item Energy trading operations simulated, not integrated with real smart meters
\end{enumerate}

\section{Future Work}

Several directions for future research emerge from this work:

\begin{enumerate}
  \item \textbf{Multi-Validator Network}: Deploy and evaluate a geographically distributed PoA validator network
  \item \textbf{Smart Meter Integration}: Connect to real IoT smart meter infrastructure
  \item \textbf{Cross-Chain Interoperability}: Evaluate bridges to other blockchain networks
  \item \textbf{Privacy Extensions}: Implement zero-knowledge proofs for transaction privacy
  \item \textbf{Regulatory Compliance}: Integrate energy market regulatory requirements
\end{enumerate}

\section{Final Remarks}

This thesis demonstrates that blockchain technology has matured to the point where it can provide viable performance for real-time P2P energy trading applications. The GridTokenX platform, with its Proof of Authority consensus mechanism, offers a pragmatic balance between decentralization and performance suitable for enterprise microgrid deployments.

As renewable energy adoption accelerates and prosumer participation increases, platforms like GridTokenX will play a crucial role in enabling efficient, transparent, and automated energy trading in smart grid environments.
