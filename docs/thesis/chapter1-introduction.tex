\chapter{Introduction}
\label{chap:introduction}

\section{Research Background}

The global transition to renewable energy sources has fundamentally transformed the electricity sector. Solar panels, wind turbines, and battery storage systems are increasingly being deployed at residential and commercial buildings, creating a new class of energy participants known as "prosumers" – consumers who also produce energy.

Traditional centralized electricity grids were designed for one-way power flow from large generation plants to passive consumers. However, prosumer participation requires bidirectional energy flow and real-time coordination between thousands of distributed energy resources (DERs). This paradigm shift has created demand for peer-to-peer (P2P) energy trading platforms that can:

\begin{itemize}
  \item Enable direct transactions between prosumers without intermediaries
  \item Provide transparent and tamper-proof transaction records
  \item Support real-time microgrid energy balancing
  \item Integrate with smart meters and IoT devices
\end{itemize}

Blockchain technology offers a promising solution for P2P energy trading due to its inherent properties of decentralization, immutability, and programmable smart contracts. However, the performance characteristics of blockchain platforms have traditionally been a concern for real-time applications.

\section{Problem Statement}

While blockchain provides the trust and transparency required for P2P energy markets, significant challenges remain:

\begin{enumerate}
  \item \textbf{Performance Uncertainty}: Can blockchain platforms achieve the throughput and latency requirements for real-time energy trading?
  
  \item \textbf{Scalability Concerns}: How do blockchain systems scale with increasing numbers of prosumer participants?
  
  \item \textbf{Trust Premium}: What is the performance cost of using blockchain compared to centralized alternatives?
  
  \item \textbf{Benchmark Methodology}: How should blockchain performance be evaluated using standardized benchmarks?
\end{enumerate}

\section{Research Objectives}

This thesis aims to:

\begin{enumerate}
  \item \textbf{Develop GridTokenX}: Design and implement a Solana-based blockchain platform for P2P energy trading using Proof of Authority (PoA) consensus.
  
  \item \textbf{Adapt TPC Benchmarks}: Apply the TPC "blockchainification" methodology to adapt TPC-C, TPC-E, TPC-H, and Smallbank benchmarks for blockchain evaluation.
  
  \item \textbf{Evaluate Performance}: Conduct comprehensive performance analysis measuring throughput, latency, and scalability.
  
  \item \textbf{Quantify Trust Premium}: Establish and measure the Trust Premium metric comparing blockchain to centralized baselines.
  
  \item \textbf{Compare with Literature}: Position GridTokenX performance against existing blockchain platforms.
\end{enumerate}

\section{Research Scope}

This research focuses on:

\begin{itemize}
  \item Performance evaluation of Solana-based blockchain with PoA consensus
  \item TPC benchmark adaptation for energy trading operations
  \item Simulation-based testing using LiteSVM and local validator
  \item Comparison with published results from Hyperledger Fabric and Ethereum
\end{itemize}

The following aspects are outside the scope of this research:

\begin{itemize}
  \item Integration with physical smart meter infrastructure
  \item Regulatory compliance for specific energy markets
  \item Economic analysis of energy pricing mechanisms
  \item Multi-validator geographic distribution testing
\end{itemize}

\section{Research Contributions}

This thesis makes the following contributions:

\begin{enumerate}
  \item A \textbf{TPC benchmark adaptation framework} for blockchain performance evaluation
  \item The \textbf{GridTokenX platform} with five integrated smart contracts
  \item \textbf{Trust Premium metric} for quantifying the cost of decentralization
  \item \textbf{Comprehensive performance analysis} with 21,101 tpmC demonstrated
  \item \textbf{Scalability validation} showing linear scaling to 200 concurrent users
\end{enumerate}

\section{Thesis Organization}

The remainder of this thesis is organized as follows:

\begin{itemize}
  \item \textbf{Chapter 2: Literature Review} examines blockchain technology, P2P energy trading, and benchmark methodologies.
  
  \item \textbf{Chapter 3: Methodology} describes the TPC benchmark adaptation and experimental setup.
  
  \item \textbf{Chapter 4: Results} presents the performance evaluation results.
  
  \item \textbf{Chapter 5: Discussion} analyzes findings and implications.
  
  \item \textbf{Chapter 6: Conclusion} summarizes contributions and future work.
\end{itemize}
