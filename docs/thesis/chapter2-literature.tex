\chapter{Literature Review}
\label{chap:literature}

\section{Blockchain Technology}

\subsection{Fundamentals}

Blockchain is a distributed ledger technology that maintains a continuously growing list of records, called blocks, which are cryptographically linked and secured. First introduced by Nakamoto (2008) for Bitcoin, blockchain technology has evolved to support programmable smart contracts and diverse consensus mechanisms.

Key properties of blockchain include:

\begin{itemize}
  \item \textbf{Decentralization}: No single point of control or failure
  \item \textbf{Immutability}: Historical records cannot be altered
  \item \textbf{Transparency}: All participants can verify transactions
  \item \textbf{Programmability}: Smart contracts enable automated execution
\end{itemize}

\subsection{Consensus Mechanisms}

Consensus mechanisms ensure agreement among distributed nodes:

\begin{table}[htbp]
\centering
\caption{Comparison of Consensus Mechanisms}
\label{tab:consensus-comparison}
\begin{tabular}{|l|l|l|l|}
\hline
\textbf{Mechanism} & \textbf{Throughput} & \textbf{Finality} & \textbf{Energy Use} \\
\hline
Proof of Work (PoW) & Low & Probabilistic & High \\
Proof of Stake (PoS) & Medium & Probabilistic & Low \\
Proof of Authority (PoA) & High & Deterministic & Low \\
PBFT & High & Immediate & Low \\
\hline
\end{tabular}
\end{table}

\textbf{Proof of Authority (PoA)}, used by GridTokenX, provides high throughput with deterministic finality, making it suitable for enterprise and permissioned blockchain applications.

\subsection{Smart Contract Platforms}

\subsubsection{Ethereum}

Ethereum introduced the concept of a "world computer" with Turing-complete smart contracts. Post-merge (PoS), Ethereum achieves approximately 30 TPS with 12-second block times. Gas fees remain a concern for high-frequency applications.

\subsubsection{Hyperledger Fabric}

Hyperledger Fabric is an enterprise-focused blockchain with pluggable consensus. Studies at TPC Technology Conferences have reported 200-400 TPS for TPC-C style workloads, with latencies in the 150-350ms range.

\subsubsection{Solana}

Solana uses a novel Proof of History (PoH) combined with Tower BFT consensus, achieving thousands of TPS. Its architecture supports high-frequency applications with sub-second finality.

\section{Peer-to-Peer Energy Trading}

\subsection{Prosumer Economy}

The rise of distributed energy resources (DERs) has created prosumers – participants who both produce and consume energy. P2P energy trading enables:

\begin{itemize}
  \item Direct transactions between neighbors
  \item Local energy consumption optimization
  \item Reduced transmission losses
  \item Community energy resilience
\end{itemize}

\subsection{Blockchain for Energy Trading}

Blockchain applications in energy trading include:

\begin{itemize}
  \item \textbf{Brooklyn Microgrid} (Mengelkamp et al., 2018): First commercial P2P energy trading pilot
  \item \textbf{Power Ledger}: Australian platform for renewable energy trading
  \item \textbf{Grid+}: Ethereum-based retail energy platform
\end{itemize}

Research by Tushar et al. (2020) provides a comprehensive survey of P2P trading mechanisms, identifying key challenges in scalability and real-time settlement.

\subsection{Requirements for Energy Trading Platforms}

Based on literature, key requirements include:

\begin{enumerate}
  \item \textbf{Low Latency}: Sub-second response for real-time pricing
  \item \textbf{High Throughput}: Support for frequent microtransactions
  \item \textbf{Scalability}: Growth with prosumer adoption
  \item \textbf{Interoperability}: Integration with smart meters and grid systems
\end{enumerate}

\section{Database Benchmarking}

\subsection{TPC Benchmarks}

The Transaction Processing Performance Council (TPC) develops standardized benchmarks for database systems:

\begin{itemize}
  \item \textbf{TPC-C}: OLTP benchmark simulating warehouse/order processing
  \item \textbf{TPC-E}: Financial market trading simulation
  \item \textbf{TPC-H}: Decision support/analytics queries
\end{itemize}

TPC benchmarks provide:
\begin{itemize}
  \item Standardized workload definitions
  \item Rigorous reporting requirements
  \item Fair comparison methodology
\end{itemize}

\subsection{Blockchain Benchmarking}

\subsubsection{BLOCKBENCH}

Dinh et al. (2017) introduced BLOCKBENCH, a framework for analyzing private blockchains. BLOCKBENCH adapts YCSB and Smallbank workloads for blockchain evaluation.

\subsubsection{TPC "Blockchainification"}

Recent TPCTC papers have explored adapting TPC benchmarks for blockchain:

\begin{itemize}
  \item Ruan et al. (2023): TPC-C on Hyperledger Fabric achieving 200 TPS
  \item Schema transformation for blockchain storage
  \item MVCC conflict handling in distributed ledgers
\end{itemize}

This methodology guides our TPC-C adaptation for GridTokenX.

\subsection{Performance Metrics}

Standard metrics for blockchain benchmarking:

\begin{table}[htbp]
\centering
\caption{Blockchain Performance Metrics}
\label{tab:metrics}
\begin{tabular}{|l|l|}
\hline
\textbf{Metric} & \textbf{Description} \\
\hline
TPS & Transactions per second \\
Latency & Time from submission to confirmation \\
Throughput & Sustained transaction rate \\
Finality & Time until transaction is irreversible \\
\hline
\end{tabular}
\end{table}

\section{Research Gap}

While existing research addresses blockchain performance, gaps remain:

\begin{enumerate}
  \item Limited TPC benchmark adaptation for Solana-based platforms
  \item Lack of Trust Premium quantification
  \item Insufficient scalability analysis for P2P energy trading
  \item Need for PoA consensus performance evaluation
\end{enumerate}

This thesis addresses these gaps through comprehensive GridTokenX evaluation using adapted TPC benchmarks.
