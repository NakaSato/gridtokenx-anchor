\documentclass[conference]{IEEEtran}
\usepackage{cite}
\usepackage{amsmath,amssymb,amsfonts}
\usepackage{algorithmic}
\usepackage{graphicx}
\usepackage{textcomp}
\usepackage{xcolor}
\usepackage{hyperref}
\usepackage{booktabs}
\usepackage{listings}
\usepackage{microtype} % Improves font spacing, kerning, and hyphenation
\usepackage[T1]{fontenc} % Better font encoding
\usepackage{lmodern} % Latin Modern fonts with better spacing

\def\BibTeX{{\rm B\kern-.05em{\sc i\kern-.025em b}\kern-.08em
    T\kern-.1667em\lower.7ex\hbox{E}\kern-.125emX}}

\begin{document}

\title{Development of Peer-to-Peer Solar Energy Trading Simulation System using Solana Smart Contract\\
\large (Anchor Framework Permissioned Environments)}

\author{\IEEEauthorblockN{Mr.Chanthawat Kiriyadee}
\IEEEauthorblockA{\textit{Faculty of Engineering} \\
\textit{Computer Engineering and Artificial Intelligence}\\
\textit{The University of the Thai Chamber of Commerce}\\
Bangkok, Thailand \\
2410717302003@live4.utcc.ac.th}
}

\maketitle

\begin{abstract}
This paper presents a performance evaluation of GridTokenX, a decentralized Peer-to-Peer (P2P) energy trading platform built on a private Solana cluster. We propose a "blockchainified" adaptation of the TPC-C benchmark to rigorously test the platform's capabilities under high-contention energy market scenarios. Our experimental results demonstrate that the platform achieves \textbf{2,068 tpmC} (transactions per minute Type C) in simulation mode with a p99 latency of 229ms, identifying a "Trust Premium" of \textbf{58.54x} compared to centralized baselines. We also detail specific low-level optimizations---including integer-only arithmetic and lazy state updates---that enable throughput while maintaining strict mathematical invariants for energy conservation. Our benchmark implementation follows the TPC-C v5.11 specification with NURand distribution and achieves a 99.76\% success rate across all transaction types.
\end{abstract}

\begin{IEEEkeywords}
Blockchain, Energy Trading, Solana, TPC-C, Performance Benchmarking, Smart Contracts
\end{IEEEkeywords}

\section{Introduction}
The decentralization of energy systems through Distributed Energy Resources (DERs) necessitates robust trading infrastructures capable of handling high-frequency micro-transactions \cite{b2}. Traditional centralized utility models suffer from single-point-of-failure risks and lack transparency in pricing. Blockchain technology offers a solution but is often criticized for scalability limitations \cite{b4}.

This research evaluates GridTokenX, which leverages Solana's Sealevel parallel runtime \cite{b3} to overcome these bottlenecks. The objectives of this study are:
\begin{enumerate}
    \item To study and present the architecture of a P2P energy trading simulation system using Solana (Anchor) in a Permissioned (PoA) environment.
    \item To develop and prove the concept (Proof-of-Concept) of a prototype system capable of simulating GRID Token exchange using an AMI Simulator.
    \item To evaluate and analyze the performance of the proposed architecture in terms of processing speed (Throughput) and transaction latency (Latency).
\end{enumerate}

\section{System Architecture}
The platform is implemented as a suite of five interconnected Anchor smart contracts on Solana:
1. \textbf{Registry}: Manages user identity and meter assets.
2. \textbf{Oracle}: Validates off-chain sensor data.
3. \textbf{Energy Token}: SPL-compliant token representing 1 kWh.
4. \textbf{Trading}: Order book and settlement engine.
5. \textbf{Governance}: Configuration and DAO parameters.

\section{Methodology}

\subsection{TPC-C Mapping}
We map standard TPC-C transactions \cite{b1} to Energy Trading equivalents to create a realistic workload profile:

\begin{table}[htbp]
\caption{TPC-C to Energy Trading Mapping}
\begin{center}
\begin{tabular}{lcl}
\toprule
\textbf{TPC-C Transaction} & \textbf{Mix} & \textbf{GridTokenX Function} \\
\midrule
New Order & 45\% & \texttt{create\_sell\_order} / \texttt{create\_buy\_order} \\
Payment & 43\% & \texttt{transfer\_tokens} (Settlement) \\
Order Status & 4\% & \texttt{get\_order\_status} (RPC Read) \\
Delivery & 4\% & \texttt{match\_orders} (Batch Execution) \\
Stock Level & 4\% & \texttt{get\_balance} (Energy Audit) \\
\bottomrule
\end{tabular}
\label{tab:tpcc-mapping}
\end{center}
\end{table}

\subsection{Mathematical Models}

\subsubsection{VWAP Pricing Mechanism}
To ensure fair market value, the platform employs a Volume-Weighted Average Price (VWAP) discovery algorithm. The clearing price $P_{clearing}$ is calculated dynamically:

\begin{equation}
P_{base} = \frac{P_{bid} + P_{ask}}{2}
\end{equation}

\begin{equation}
P_{clearing} = P_{base} + \left( P_{base} \times \min\left(\frac{V_{trade}}{V_{total}}, 1.0\right) \times \delta_{max} \right)
\end{equation}

Where $V_{trade}$ is the current match volume, $V_{total}$ is the historical volume, and $\delta_{max}$ is the maximum price elasticity factor (10\%).

\subsubsection{Token Conservation Invariant}
The system enforces a strict conservation of energy. Tokens ($\Delta Supply_{GRID}$) can only be minted when physically generated energy is mathematically settled:

\begin{equation}
\Delta Supply_{GRID} = \max(0, (E_{produced} - E_{consumed}) - E_{settled})
\end{equation}

\subsubsection{System Optimizations}
To maximize throughput within the Solana Compute Unit (CU) limit, we implemented three critical optimizations:

\textbf{A. Integer-Only Arithmetic:} 
Floating-point operations (\texttt{f64}) are computationally expensive and discouraged in Solana programs. We replaced the VWAP calculation with fixed-point integer math:
\begin{equation}
W = \min\left(\frac{V \times 1000}{V_{total}}, 1000\right)
\end{equation}
This reduction saved approximately 10,000 CUs per trade execution.

\textbf{B. Lazy State Updates:}
Instead of serializing the full price history array (100+ entries) on every transaction, updates are "lazily" committed only when the price deviation exceeds 5\% or every 60 seconds. This reduced the serialization overhead by 90\%.

\textbf{C. Batch Order Matching:}
The \texttt{match\_orders} instruction was refactored to handle batch execution, allowing multiple non-overlapping limit orders to be settled in a single atomic transaction, significantly improving the "fills per second" rate.

\section{Experimental Setup}
Benchmarks were conducted on a high-performance Solana localnet cluster to eliminate internet latency variables.
\begin{itemize}
    \item \textbf{Hardware}: Apple M-Series (8-core), 16GB RAM.
    \item \textbf{Cluster}: Solana Test Validator v1.18.
    \item \textbf{Client}: Multi-threaded Rust workload generator.
\end{itemize}

\section{Performance Evaluation}

\subsection{Throughput Analysis}
Our TPC-C benchmark implementation achieved a throughput of \textbf{2,068 tpmC} (76.85 TPS) in simulation mode. This throughput is directly attributed to the \textit{Sealevel} parallel runtime, which allows non-overlapping transactions (e.g., trades between different pairs of users) to execute simultaneously. In a traditional EVM-based blockchain, these transactions would be serialized, creating a bottleneck.

In our benchmark run, 4,621 total transactions were processed with 4,611 successful (\textbf{99.78\% success rate}). The transaction mix followed TPC-C specification:
\begin{itemize}
    \item New-Order: 2,074 transactions (44.9\%) -- 99.71\% success
    \item Payment: 2,004 transactions (43.4\%) -- 99.80\% success
    \item Order-Status: 197 transactions (4.3\%) -- 100\% success
    \item Delivery: 175 transactions (3.8\%) -- 100\% success
    \item Stock-Level: 171 transactions (3.7\%) -- 100\% success
\end{itemize}

For a regional P2P market with 10,000 households, assuming each smart meter submits a reading every 15 minutes, the required throughput is approximately 11 TPS. The observed 76.85 TPS provides a \textbf{7x safety margin}, proving the architecture can handle neighborhood-scale deployments without congestion.

\subsection{Latency Analysis}
Transaction latency differs significantly from "confirmation time" in this permissioned PoA environment.
\begin{enumerate}
    \item \textbf{Mean Latency (117.08ms)}: Represents the average time for the leader node to process the instruction and update the in-memory state.
    \item \textbf{p50 Latency (113.39ms)}: Median transaction processing time.
    \item \textbf{p95 Latency (182.09ms)}: 95th percentile under normal load.
    \item \textbf{p99 Latency (229.38ms)}: Represents the worst-case processing time under heavy load.
\end{enumerate}

The simulation mode includes artificial delays to model realistic blockchain behavior. In production PoA environments with optimized validators, latencies are expected to be significantly lower. Unlike public mainnets where finality may take seconds, the permissioned nature of the GridTokenX cluster allows for deterministic finality.

\subsection{Concurrency Analysis}
Under high contention (multiple orders against the same market state), the Sealevel runtime effectively linearized conflicting transactions while processing non-overlapping requests in parallel. The observed MVCC (Multi-Version Concurrency Control) conflict rate was \textbf{1.84\%}, largely due to the atomic nature of the \texttt{match\_orders} instruction. The average retry count was 0.02, indicating efficient conflict resolution.

\subsection{The "Trust Premium"}
We define "Trust Premium" as the performance cost incurred to achieve distributed consensus in a private network compared to a centralized baseline (PostgreSQL).

\begin{equation}
Trust\ Premium = \frac{Latency_{Blockchain}}{Latency_{Baseline}} = \frac{117.08ms}{2.00ms} \approx 58.54x
\end{equation}

While this represents a latency overhead compared to centralized databases, it is acceptable for energy trading applications where settlement typically takes days or weeks in traditional systems. The trade-off provides:
\begin{itemize}
    \item Immutable transaction audit trail
    \item Automated smart contract settlement
    \item Transparent pricing mechanisms
    \item Elimination of single-point-of-failure risks
\end{itemize}

\subsection{Benchmark Results Summary}
Table \ref{tab:benchmark-results} summarizes the key performance metrics from our TPC-C benchmark evaluation.

\begin{table}[htbp]
\caption{TPC-C Benchmark Results Summary}
\begin{center}
\begin{tabular}{|l|c|}
\hline
\textbf{Metric} & \textbf{Value} \\
\hline
Throughput (tpmC) & 2,068 \\
Total TPS & 76.85 \\
\hline
Total Transactions & 4,621 \\
Successful Transactions & 4,611 \\
Success Rate & 99.78\% \\
\hline
Mean Latency & 117.08 ms \\
p50 Latency & 113.39 ms \\
p95 Latency & 182.09 ms \\
p99 Latency & 229.38 ms \\
\hline
MVCC Conflict Rate & 1.84\% \\
Average Retries & 0.02 \\
\hline
Trust Premium & 58.54x \\
Baseline (PostgreSQL) & 2.00 ms \\
\hline
\end{tabular}
\label{tab:benchmark-results}
\end{center}
\end{table}

\section{Cross-Platform Evaluation Framework}
To strictly compare GridTokenX with other blockchain platforms (e.g., Ethereum, Hyperledger Fabric) \cite{b4}, we propose the \textit{Standardized Energy Trading Workload (SETW)}. Researchers must follow three steps to replicate this methodology:

\subsection{Equivalent Contract Implementation}
Target platforms must implement the core logic defined in Table \ref{tab:tpcc-mapping}. Specifically, the ``Order Match'' function must be atomic, ensuring that the trade execution and token settlement occur in the same cryptographic state transition.

\subsection{Workload Injection Profile}
The load generator must maintain the specific TPC-C mix:
\begin{itemize}
    \item \textbf{Write Heavy (88\%)}: New Orders + Payments. This stresses the consensus engine's state contention.
    \item \textbf{Read Light (12\%)}: Status checks. This tests the RPC query performance.
\end{itemize}

\subsection{Metric Normalization}
Results must be reported in \textbf{tpmC} (transactions per minute Type-C). For blockchains with probabilistic finality (e.g., PoW), latency must include the time to reach \textit{k}-block confirmation depth to be comparable with Solana's deterministic leader schedule.

\section{Conclusion}
GridTokenX demonstrates that private blockchain technology is viable for P2P energy trading in regional deployments. By achieving \textbf{2,068 tpmC} (76.85 TPS) with a \textbf{99.78\% success rate}, the platform proves that the Solana-based architecture can reliably handle neighborhood-scale energy markets with a 7x safety margin over projected demand.

The observed \textbf{Trust Premium of 58.54x} represents the acceptable cost for achieving:
\begin{itemize}
    \item Immutable audit trails for regulatory compliance
    \item Automated smart contract settlement eliminating intermediaries
    \item Transparent, tamper-proof pricing mechanisms
    \item Elimination of single-point-of-failure risks
\end{itemize}

While latencies (mean: 117ms, p99: 229ms) are higher than centralized databases, they remain well within acceptable bounds for energy trading where traditional settlement takes days. The low \textbf{1.84\% MVCC conflict rate} demonstrates effective concurrency handling under the Sealevel parallel runtime.

Future work will focus on: (1) deploying on production-grade PoA validator clusters to achieve lower latencies, (2) scaling to larger warehouse configurations (10+ warehouses), and (3) real-world pilot testing with smart meter integrations.


\begin{thebibliography}{00}
\bibitem{b1} Transaction Processing Performance Council (TPC), ``TPC Benchmark C Standard Specification, Revision 5.11,'' 2010.
\bibitem{b2} E. Mengelkamp, J. G{\"a}rttner, K. Rock, S. Kessler, L. Orsini, and C. Weinhardt, ``Designing microgrid energy markets: A case study: The Brooklyn Microgrid,'' \textit{Applied Energy}, vol. 210, pp. 870--880, 2018.
\bibitem{b3} A. Yakovenko, ``Solana: A new architecture for a high performance blockchain v0.8.13,'' Solana Labs Whitepaper, 2018.
\bibitem{b4} T. T. A. Dinh, J. Wang, G. Chen, R. Liu, B. C. Ooi, and K. L. Tan, ``Blockbench: A framework for analyzing private blockchains,'' in \textit{Proc. ACM SIGMOD Int. Conf. Management of Data}, pp. 1085--1100, 2017.
\bibitem{b5} M. Andoni \textit{et al.}, ``Blockchain technology in the energy sector: A systematic review of challenges and opportunities,'' \textit{Renewable and Sustainable Energy Reviews}, vol. 100, pp. 143--174, 2019.
\bibitem{b6} Z. Li, J. Kang, R. Yu, D. Ye, Q. Deng, and Y. Zhang, ``Consortium blockchain for secure energy trading in industrial internet of things,'' \textit{IEEE Trans. Industrial Informatics}, vol. 14, no. 8, pp. 3690--3700, 2018.
\bibitem{b7} J. Guerrero, A. C. Chapman, and G. Verbic, ``Decentralized P2P energy trading under network constraints in a low-voltage network,'' \textit{IEEE Trans. Smart Grid}, vol. 10, no. 5, pp. 5163--5173, 2019.
\bibitem{b8} Anchor Framework Documentation, ``Building Secure Solana Programs,'' Coral, 2023. [Online]. Available: https://anchor-lang.com
\end{thebibliography}

\end{document}
