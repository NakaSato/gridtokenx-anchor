\documentclass[11pt,a4paper]{article}
\usepackage[utf8]{inputenc}
\usepackage[T1]{fontenc}
\usepackage{times}
\usepackage{geometry}
\geometry{margin=1in}
\usepackage{graphicx}
\usepackage{amsmath,amssymb}
\usepackage{booktabs}
\usepackage{hyperref}
\usepackage{xcolor}
\usepackage{listings}
\usepackage{fancyhdr}

% Colors
\definecolor{solana}{RGB}{20,241,149}
\definecolor{darkblue}{RGB}{0,51,102}

% Header/Footer
\pagestyle{fancy}
\fancyhf{}
\fancyhead[L]{\textit{GridTokenX White Paper}}
\fancyhead[R]{\textit{December 2025}}
\fancyfoot[C]{\thepage}

% Hyperlinks
\hypersetup{
    colorlinks=true,
    linkcolor=darkblue,
    urlcolor=darkblue,
    citecolor=darkblue
}

\begin{document}

% Title Page
\begin{titlepage}
\centering
\vspace*{2cm}

{\Huge\bfseries GridTokenX}\\[0.5cm]
{\LARGE Blockchain-Based Peer-to-Peer Solar Energy Trading Platform}\\[0.3cm]
{\large White Paper v2.0}\\[2cm]

{\Large\textit{Development of Peer-to-Peer Solar Energy Trading\\Simulation System using Solana Smart Contract\\(Anchor Framework Permissioned Environments)}}\\[2cm]

{\large Mr. Chanthawat Kiriyadee}\\[0.3cm]
{\normalsize Faculty of Engineering\\Computer Engineering and Artificial Intelligence\\The University of the Thai Chamber of Commerce\\Bangkok, Thailand}\\[0.3cm]
{\small 2410717302003@live4.utcc.ac.th}\\[3cm]

{\large December 2025}\\[1cm]

\vfill
{\small \textbf{CONFIDENTIAL} -- Private Network (PoA) Design Document}
\end{titlepage}

% Table of Contents
\tableofcontents
\newpage

% Abstract
\section*{Abstract}
\addcontentsline{toc}{section}{Abstract}

This white paper presents GridTokenX, a conceptual design for a decentralized Peer-to-Peer (P2P) solar energy trading simulation platform operating on a private Solana network. The platform enables prosumers to tokenize surplus solar energy production into GRID tokens (1 kWh = 1 GRID) and trade directly with consumers through an automated order book mechanism.

Our implementation leverages the Anchor framework to deploy five interconnected smart contracts: Registry (user/meter management), Oracle (data validation), Energy Token (SPL-compliant minting), Trading (order book settlement), and Governance (configuration). Performance benchmarks demonstrate 21,378 tpmC (356 TPS) with 11ms average latency and a ``Trust Premium'' of 5.67x compared to centralized databases.

The objectives of this research are: (1) to study and present the architecture of a P2P energy trading simulation using Solana in a Permissioned (PoA) environment; (2) to develop a Proof-of-Concept capable of simulating GRID Token exchange with an AMI Simulator; and (3) to evaluate the performance in terms of Throughput and Latency.

\textbf{Keywords:} Blockchain, Solana, P2P Energy Trading, Smart Contracts, Tokenization, TPC-C

\newpage

% Chapter 1: Executive Summary
\section{Executive Summary}

\subsection{Problem Statement}

The traditional energy market faces fundamental challenges limiting distributed renewable energy adoption:

\begin{itemize}
    \item \textbf{Centralized Intermediaries}: Energy flows through utilities with high fees and limited transparency
    \item \textbf{Lack of Direct Trading}: Prosumers cannot sell directly to neighbors
    \item \textbf{Settlement Inefficiencies}: Monthly billing cycles and delayed payments
    \item \textbf{Trust and Verification}: Reliance on utility meters with limited audit capabilities
\end{itemize}

\subsection{Proposed Solution}

GridTokenX addresses these challenges through a blockchain-based P2P energy trading simulation platform designed for private/permissioned Solana networks:

\begin{itemize}
    \item \textbf{Tokenization}: 1 kWh = 1 GRID token (SPL-compliant, 9 decimals)
    \item \textbf{Direct P2P Trading}: Order book matching without intermediaries
    \item \textbf{Real-time Settlement}: Atomic trade execution with instant finality
    \item \textbf{Verified Green Energy}: On-chain ERC certificates
\end{itemize}

\subsection{Technical Innovation}

\begin{table}[htbp]
\caption{Platform Comparison}
\centering
\begin{tabular}{lccc}
\toprule
\textbf{Criterion} & \textbf{Solana (Private)} & \textbf{Ethereum} & \textbf{Polygon} \\
\midrule
Transaction Speed & 11ms & 12-15s & 2s \\
Cost per TX & \$0.00025 & \$1-50 & \$0.01 \\
TPS Capacity & 356+ (tested) & 15-30 & 7,000 \\
Finality & Deterministic & 6 blocks & 256 blocks \\
\bottomrule
\end{tabular}
\end{table}

\newpage

% Chapter 2: Business Model
\section{Business Model}

\subsection{Revenue Streams}

The platform generates revenue through three primary channels:

\begin{enumerate}
    \item \textbf{Transaction Fees (60\% of revenue)}
    \begin{itemize}
        \item Trade fee: 0.25\% (split between buyer/seller)
        \item Settlement fee: 0.1\% (seller)
    \end{itemize}
    
    \item \textbf{Certificate Fees (25\% of revenue)}
    \begin{itemize}
        \item ERC Issuance: 5 GRID per certificate
        \item ERC Validation: 2 GRID per approval
    \end{itemize}
    
    \item \textbf{Premium Services (15\% of revenue)}
    \begin{itemize}
        \item API Access: 100 GRID/month
        \item Premium Analytics: 50 GRID/month
    \end{itemize}
\end{enumerate}

\subsection{Compute Economics (Private Network)}

Since GridTokenX operates on a private/permissioned Solana network, compute costs differ significantly from public mainnet:

\begin{table}[htbp]
\caption{Compute Unit Cost Breakdown}
\centering
\begin{tabular}{lccc}
\toprule
\textbf{Instruction} & \textbf{Est. CU} & \textbf{Public Cost} & \textbf{Private Cost} \\
\midrule
\texttt{create\_sell\_order} & 50,000 & \$0.025 & \$0.001 \\
\texttt{create\_buy\_order} & 45,000 & \$0.023 & \$0.001 \\
\texttt{match\_orders} & 80,000 & \$0.040 & \$0.002 \\
\texttt{submit\_reading} & 30,000 & \$0.015 & \$0.0005 \\
\texttt{mint\_tokens} & 60,000 & \$0.030 & \$0.001 \\
\bottomrule
\end{tabular}
\end{table}

\textbf{Private Network Advantage:} No validator fees results in 95\%+ cost reduction vs public Solana.

\newpage

% Chapter 3: System Architecture
\section{System Architecture}

\subsection{Four-Layer Model}

The platform architecture consists of four distinct layers:

\begin{enumerate}
    \item \textbf{Presentation Layer}: Web/Mobile clients, user interface
    \item \textbf{Application Layer}: API Gateway, WebSocket server, event processors
    \item \textbf{Data Layer}: PostgreSQL (off-chain), Redis (cache), Solana RPC
    \item \textbf{Blockchain Layer}: Anchor programs, SPL Token, System Program
\end{enumerate}

\subsection{Smart Contract Programs}

Five interconnected Anchor programs form the blockchain layer:

\begin{table}[htbp]
\caption{Program Architecture}
\centering
\begin{tabular}{lll}
\toprule
\textbf{Program} & \textbf{Purpose} & \textbf{Key Functions} \\
\midrule
Registry & User/Meter Management & \texttt{register\_user}, \texttt{register\_meter} \\
Oracle & Data Validation & \texttt{update\_price}, \texttt{submit\_reading} \\
Energy Token & Token Operations & \texttt{mint\_from\_production}, \texttt{burn} \\
Trading & Marketplace & \texttt{create\_sell\_order}, \texttt{match\_orders} \\
Governance & Configuration & \texttt{issue\_erc}, \texttt{validate\_erc} \\
\bottomrule
\end{tabular}
\end{table}

\subsection{Program Relationships}

Programs interact through Cross-Program Invocation (CPI):
\begin{itemize}
    \item Registry $\rightarrow$ Energy Token: Mint tokens from settled production
    \item Trading $\rightarrow$ SPL Token: Escrow and settlement transfers
    \item Oracle $\rightarrow$ Registry: Submit verified meter readings
\end{itemize}

\newpage

% Chapter 4: Token Economics
\section{Token Economics}

\subsection{GRID Token Specification}

\begin{itemize}
    \item \textbf{Standard}: SPL Token (Solana Program Library)
    \item \textbf{Decimals}: 9 (divisible to nano-tokens)
    \item \textbf{Backing}: 1 GRID = 1 kWh verified energy production
    \item \textbf{Supply}: Elastic (minted on production, burned on consumption)
\end{itemize}

\subsection{Token Conservation Invariant}

The system enforces strict conservation of energy. Tokens can only be minted when physically generated energy is mathematically settled:

\begin{equation}
\Delta Supply_{GRID} = \max(0, (E_{produced} - E_{consumed}) - E_{settled})
\end{equation}

This prevents double-spending by tracking the $E_{settled}$ accumulator per meter.

\subsection{VWAP Pricing Mechanism}

The clearing price is calculated using Volume-Weighted Average Price:

\begin{equation}
P_{base} = \frac{P_{bid} + P_{ask}}{2}
\end{equation}

\begin{equation}
P_{clearing} = P_{base} + \left( P_{base} \times \min\left(\frac{V_{trade}}{V_{total}}, 1.0\right) \times 0.10 \right)
\end{equation}

\newpage

% Chapter 5: Performance Evaluation
\section{Performance Evaluation}

\subsection{Benchmark Methodology}

We adapt the TPC-C benchmark for energy trading workloads:

\begin{table}[htbp]
\caption{TPC-C to Energy Trading Mapping}
\centering
\begin{tabular}{lcp{4cm}}
\toprule
\textbf{TPC-C Transaction} & \textbf{Mix} & \textbf{GridTokenX Function} \\
\midrule
New Order & 45\% & \texttt{create\_sell\_order} / \texttt{create\_buy\_order} \\
Payment & 43\% & \texttt{transfer\_tokens} \\
Order Status & 4\% & \texttt{get\_order\_status} \\
Delivery & 4\% & \texttt{match\_orders} \\
Stock Level & 4\% & \texttt{get\_balance} \\
\bottomrule
\end{tabular}
\end{table}

\subsection{Results}

\begin{table}[htbp]
\caption{Performance Benchmark Results (Local Validator)}
\centering
\begin{tabular}{lr}
\toprule
\textbf{Metric} & \textbf{Observed Value} \\
\midrule
tpmC (Transactions/min) & 21,101 \\
TPS Equivalent & 352 \\
Total Transactions & 23,848 \\
Successful Transactions & 23,817 \\
Average Latency & 11.30 ms \\
p50 Latency & 11.00 ms \\
p95 Latency & 18.00 ms \\
p99 Latency & 20.00 ms \\
Transaction Success Rate & 99.9\% \\
MVCC Conflict Rate & 1.3\% \\
\bottomrule
\end{tabular}
\end{table}

\subsection{Trust Premium}

\begin{equation}
Trust\ Premium = \frac{Latency_{Blockchain}}{Latency_{Baseline}} = \frac{11.34ms}{2.00ms} \approx 5.67x
\end{equation}

This overhead is negligible for energy trading where traditional settlement takes days.

\newpage

% Chapter 6: Security Analysis
\section{Security Analysis}

\subsection{Threat Model}

\begin{itemize}
    \item \textbf{Meter Spoofing}: Mitigated by Ed25519 signature validation
    \item \textbf{Double Spending}: Prevented by atomic settlement and PDA ownership
    \item \textbf{Front Running}: Reduced by 11ms finality window
    \item \textbf{Oracle Manipulation}: Anomaly detection at 10x threshold
\end{itemize}

\subsection{Access Control}

Program Derived Addresses (PDAs) enforce ownership:

\begin{verbatim}
seeds = [b"order", authority.key(), market.active_orders.to_le_bytes()]
\end{verbatim}

Only the original authority can modify or cancel their orders.

\newpage

% Chapter 7: Comparative Analysis
\section{Comparative Analysis}

\begin{table}[htbp]
\caption{Platform Comparison}
\centering
\begin{tabular}{lccc}
\toprule
\textbf{Platform} & \textbf{Avg Latency} & \textbf{Finality} & \textbf{Notes} \\
\midrule
GridTokenX & 11ms & <1s & Solana PoA \\
Power Ledger & 3-5s & 10s & Custom chain \\
Energy Web & 5s & 5s & PoA chain \\
WePower & 15s & 15s & Ethereum \\
\bottomrule
\end{tabular}
\end{table}

\newpage

% Chapter 8: Future Roadmap
\section{Future Roadmap}

\subsection{Development Phases}

\begin{enumerate}
    \item \textbf{Phase 1 (Q1 2025)}: Pilot launch with 100 prosumers
    \item \textbf{Phase 2 (Q2-Q4 2025)}: Scale to 10,000 users
    \item \textbf{Phase 3 (2026)}: Southeast Asia expansion
    \item \textbf{Phase 4 (2027+)}: Cross-chain integration, AI optimization
\end{enumerate}

\subsection{Performance Targets}

\begin{table}[htbp]
\caption{Performance Roadmap}
\centering
\begin{tabular}{lcccc}
\toprule
\textbf{Metric} & \textbf{v1.0} & \textbf{v2.0} & \textbf{v3.0} & \textbf{v4.0} \\
\midrule
TPS & 50 & 200 & 1,000 & 10,000+ \\
Latency (p99) & 20ms & 15ms & 10ms & 5ms \\
Users & 10K & 100K & 1M & 10M \\
\bottomrule
\end{tabular}
\end{table}

\newpage

% References
\section*{References}
\addcontentsline{toc}{section}{References}

\begin{enumerate}
\item Transaction Processing Performance Council (TPC), ``TPC Benchmark C Standard Specification, Revision 5.11,'' 2010.
\item E. Mengelkamp et al., ``Designing microgrid energy markets: A case study: The Brooklyn Microgrid,'' \textit{Applied Energy}, vol. 210, pp. 870--880, 2018.
\item A. Yakovenko, ``Solana: A new architecture for a high performance blockchain v0.8.13,'' Solana Labs Whitepaper, 2018.
\item T. T. A. Dinh et al., ``Blockbench: A framework for analyzing private blockchains,'' \textit{ACM SIGMOD}, pp. 1085--1100, 2017.
\item M. Andoni et al., ``Blockchain technology in the energy sector: A systematic review,'' \textit{Renewable and Sustainable Energy Reviews}, vol. 100, pp. 143--174, 2019.
\item Z. Li et al., ``Consortium blockchain for secure energy trading in IIoT,'' \textit{IEEE Trans. Industrial Informatics}, vol. 14, no. 8, pp. 3690--3700, 2018.
\item Anchor Framework Documentation, ``Building Secure Solana Programs,'' Coral, 2023.
\end{enumerate}

\end{document}
